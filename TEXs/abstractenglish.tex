\chapter*{Abstract}
% \pdfbookmark[1]{Abstract}{Abstract}
\thispagestyle{empty}
Density-induced fingering due to dissolution of \ce{CO2} in water is considered a mechanism for 
speleogenesis. \citet{Scherzer2017} stated this process as "Nerochytic Speleogenesis" (NERO). 
Despite having similar effects to density-driven fingering of \ce{CO2} in geological sequestration 
of greenhouse gases, it is so far not discussed as a mechanism for speleogenesis. Understanding the 
process could help in understanding karst formation, also known as karstification.\\

We developed a model in \DuMuX, also validated with \MATLAB, that considered the dissolution of calcium-carbonate 
as a source/sink term in the Navier-Stokes model. We found the concentration of \ce{CO2} at the top of 
the cave water table that leads to a concentration gradient between the region above and below the epiphreatic karst water table, 
and the flow-velocity to influence the rate of calcite dissolution. We ran simulations for different 
scenarios and compared the results to see the effects on the rate of dissolution on a 2D-domain of  size 15mm$\times$5mm. 
The numerical comparison between \DuMuX and \MATLAB showed good consistency in predicting the rate of dissolution. 
We could not consider some of the scenarios prevalent in bigger geological cave systems in our model; therefore, we could not 
predict the time scale for dissolution of calcite for such scenarios, although the model concept and numerical 
implementation are transferable to such domain. We imposed limitations in our model by switching-off the 
gravity; fixing the concentration of \ce{CO2} at the top of the cave for a simulation run, which would vary throughout the year (\cref{tab:CO2fluctuations}); and 
setting a constant density throughout the domain during simulation runs, ignoring the impact of density change of karst 
water due to addition of total inorganic carbon and calcium.\\

Expanding the model to account for the aforementioned limitations would be the next step in exploring density-induced 
fingering of \ce{CO2}-enriched water as a mechanism for speleogenesis.

\endinput