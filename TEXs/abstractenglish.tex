\chapter*{Abstract}
% \pdfbookmark[1]{Abstract}{Abstract}
\thispagestyle{empty}
Density-induced fingering due to dissolution of \ce{CO2} in water is considered a mechanism for 
speleogenesis. \citet{Scherzer2017} stated this process as "Nerochytic Speleogenesis" (NERO). 
Despite having similar effects to density-driven fingering of \ce{CO2} in geological sequestration 
of green-house gases, it is so far not discussed as a mechanism for speleogenesis. Understanding the 
process could help in understanding karst formation, also known as karstification.\\

We developed a model in \DuMuX, also validated with \MATLAB, which considered the dissolution of calcium-carbonate 
as a source/sink term in the Navier-Stokes model. We found pH of the karst water, concentration of \ce{CO2} at the 
top of the cave and flow-velocity influence the rate of calcite dissolution. We ran simulations for different 
scenarios and compared the results to see the effects on the rate of dissolution on the 2D-domain of  size 15mm$\times$5mm. 
The numerical comparison between \DuMuX and \MATLAB showed good consistency in predicting the rate of dissolution. 
We were not able to model for a real-world karstification scenario with a bigger domain; therefore, we could not 
predict the time scale for dissolution of calcite for such a scenario, although the model concept and numerical 
implementation are transferable to the bigger domain. We imposed limitations in our model by switching-off the 
gravity; fixing the concentration of \ce{CO2} at the top of the cave for a simulation run, which would vary throughout the year; and 
setting a constant density throughout the domain and simulation run, ignoring the impact of density change of karst 
water due to addition of \ce{CO2} and calcium.\\

Expanding the model to account for the aforementioned limitations would be the next step in exploring density-induced 
fingering of \ce{CO2}-enriched water as a mechanism for speleogenesis.

\endinput