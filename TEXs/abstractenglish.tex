\chapter*{Abstract}
% \pdfbookmark[1]{Abstract}{Abstract}
\thispagestyle{empty}
Density-induced fingering due to dissolution of \ce{CO2} in water is considered a mechanism for 
speleogenesis. \citet{Scherzer2017} stated this process as \say{Nerochytic Speleogenesis} (NERO). 
Despite having similar effects to density-driven fingering of \ce{CO2} in geological sequestration 
of greenhouse gases, it is so far not discussed as a mechanism for speleogenesis. Understanding the 
process could help in understanding karst formation, also known as karstification.\\

We developed a model in \DuMuX, also validated with \MATLAB, that considered the dissolution of calcium-carbonate 
as a source/sink term in the Navier-Stokes model. We found the concentration of \ce{CO2} at the top of 
the cave water table that leads to a concentration gradient between the region above and below the epiphreatic karst water table, 
and the flow-velocity to influence the rate of calcite dissolution. We ran simulations for different 
scenarios and compared the results to see the effects on the rate of calcite dissolution. The numerical comparison 
between \DuMuX and \MATLAB showed consistency in predicting the rate of calcite dissolution. \\

Modeling a real-world karstification process in a bigger geological cave system is computationally expensive 
because of its size, presence of numerous cave minerals in karst water, and complexities involved in 
solving the governing equations. Therefore, to achieve simplification we imposed a few limitations in our model. 
We modeled a 2D-domain of size [15mm$\times$5mm]; considered only total inorganic carbon and calcium as source/sink terms; and 
switched-off the gravity, fixed the concentration of \ce{CO2} at the top of the cave to a constant value, and 
set the density of karst water as a constant throughout the domain. \\

\endinput