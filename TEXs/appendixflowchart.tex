\chapter{APPENDIX: Flowcharts}\label{app:appendixA}
\thispagestyle{empty}


\section{Flowchart for the main program flow in \DuMuX}
\centering
\begin{tikzpicture}[node distance=1.7cm]

\node (main) [startstop] {Main};

\node (func3) [predefinedprocess, below of=main] {Set initial and boundary conditions for the problem};

\node (pro12) [process, below of=func3] {Initialize solution vector as per Navier-Stokes model};

\node (in1)[io, below of=pro12] {Import time loop parameters: tEnd, maxDt and dt};

\node (pro1)[process, below of=in1,yshift=-0.5cm] {Add fields in the output file for pH, $\mathrm{m^{Ca^{2+}}}$, $\mathrm{m^{TIC}}$, $\mathrm{m^{CO_3^{2-}}}$, 
$\mathrm{r_{diss}}$, and Omega to visualize later in Paraview};

\node (pro2)[process, below of=pro1,yshift=-1.0cm] {Initialize a file stream to save the above values to get a time series plot};

\node (func1) [predefinedprocess, below of=pro2, yshift=-1.0cm] {call \code{setValues} function which calculates and saves pH, 
$\mathrm{m^{Ca^{2+}}}$, $\mathrm{m^{TIC}}$, $\mathrm{m^{CO_3^{2-}}}$, $\mathrm{r_{diss}}$, and Omega in their respective vectors};

\node (conn1)[connector, below of=func1, yshift=-1.0cm] {A};

\node (conn2)[connector, right of=conn1, xshift=5cm] {B};
 
% lines
\path [line] (main) -- (func3);
\path [line] (func3) -- (pro12);
\path [line] (pro12) -- (in1);
\path [line] (in1) -- (pro1);
\path [line] (pro1) -- (pro2);
\path [line] (pro2) -- (func1);
\path [line] (func1) -- (conn1);
\path [line] (conn2) |- (func1);
\end{tikzpicture}


\begin{tikzpicture}[node distance=2cm]

\node (conn2)[connector] {A};

\node (func4) [predefinedprocess, below of=conn2, yshift=-1.0cm] {Solve the governing equations; call \code{reactionSource} 
function to add source/sink terms (only for the cells at the boundary)};

\node (conn3)[connector, right of=conn2, xshift=5cm] {B};

\node (pro6)[process, below of=func4, yshift=-1.7cm] {get values of pH, $\mathrm{m^{Ca^{2+}}}$, $\mathrm{m^{TIC}}$, $\mathrm{m^{CO_3^{2-}}}$, 
$\mathrm{r_{diss}}$, and Omega by calling their respective getter function};

\node (pro7)[process, below of=pro6] { write the output values to a file};

\node(dec2) [decision, below of=pro7, yshift=-2.0cm, text width= 4.2cm] {check for the end of simulation};

\node(pro9) [process, below of=dec2, yshift=-2.0cm] {close the file stream};

\node(end)[startstop, below of=pro9] {End};

\path [line] (conn2) -- (func4);
\path [line] (func4) -- (pro6);
\path [line] (pro6) -- (pro7);
\path [line] (pro7) -- (dec2);
\path [line] (dec2) -- node[anchor=east] {yes} (pro9);
\path [line] (pro9) -- (end);
\path [line] (dec2) -| node[xshift=-0.5cm, yshift=0.3cm]{no} (conn3);
\end{tikzpicture}

\section{Flowchart for Initial and Boundary conditions in \DuMuX}
\begin{tikzpicture}[node distance=2cm]
\node (func1) [predefinedprocess] {Initial and Boundary conditions};

\node (dec1) [decision, below of=func1, yshift=-1.5cm, text width= 2.6cm] {if \code{onTopBoundary}?};
\node (pro1) [description, right of=dec1, xshift=5cm] {set Dirichlet conditions for velocities};
\node (dec2) [decision, below of=pro1, yshift=-0.8cm, text width = 2cm] {if Open system?};
\node (pro2) [description, below of=dec2, yshift=-0.8cm] {set Dirichlet conditions for calcium and TIC concentrations};
\node (pro3) [description, right of=dec2, xshift=2.0cm, text width= 3.0cm] {set Neumann no flow for calcium and TIC fluxes};
\node (return1) [description, below of=pro2,text width= 3.0cm] {return \code{values}};


\node (dec3) [decision, below of=dec1, yshift=-8cm, text width= 3.2cm] {if \code{onBottomBoundary}?};
\node (pro4) [description, right of=dec3, xshift=5cm] {set Dirichlet conditions for pressure};
\node (dec4) [decision, below of=pro4, yshift=-0.8cm, text width = 2cm] {if Open system?};
\node (pro5) [description, below of=dec4, yshift=-0.8cm] {set Outflow for calcium and TIC fluxes};
\node (pro6) [description, right of=dec4, xshift=2.0cm, text width= 3.0cm] {set Neumann no flow for calcium and TIC fluxes};
\node (return2) [description, below of=pro5, text width= 3.0cm] {return \code{values}};

\node (conn1) [connector, below of=dec3, yshift=-6cm] {A};

% lines
\path [line] (func1) -- (dec1);
\path [line] (dec1) -- node[anchor=south]{yes} (pro1);
\path [line] (pro1) -- (dec2);
\path [line] (dec2) -- node[anchor=east]{yes} (pro2);
\path [line] (dec2) -- node[anchor=south]{no} (pro3);
\path [line] (pro2) -- (return1);
\path [line] (pro3) |- (return1);

\path [line] (dec1) -- node[anchor=east]{no}  (dec3);
\path [line] (dec3) -- node[anchor=south]{yes} (pro4);
\path [line] (pro4) -- (dec4);
\path [line] (dec4) -- node[anchor=east]{yes} (pro5);
\path [line] (dec4) -- node[anchor=south]{no} (pro6);
\path [line] (pro6) |- (return2);
\path [line] (pro5) -- (return2);
\path [line] (dec3) -- node[anchor=east]{no}  (conn1);

\end{tikzpicture}


\begin{tikzpicture}[node distance=2cm]
\node (conn1) [connector] {A};
\node (dec1) [decision, below of=conn1, text width= 2.8cm, yshift=-1.0cm] {if \code{onLeftBoundary}?};
\node (pro1) [description, right of=dec1, xshift=6cm] {set Dirichlet conditions for velocities};
\node (func1) [predefinedprocess, below of=pro1, text width=5cm, yshift=-0.5cm] {set Neumann conditions for calcium and TIC 
fluxes by calling \code{reactionSource} function};
\node (return1) [description, below of=func1, text width= 3.0cm, yshift=-0.5cm] {return \code{values}};

\node (dec2) [decision, below of=dec1, yshift=-5.0cm, text width= 3.0cm] {if \code{onRightBoundary}?};
\node (pro2) [description, right of=dec2, xshift=6cm] {set symmetric property};
\node (return2) [description, below of=pro2, text width= 3.0cm] {return \code{values}};

\node (return3) [description, below of=dec2, text width= 3.0cm, yshift=-2cm] {return \code{values}};

% lines
\path [line] (conn1) -- (dec1);
\path [line] (dec1) --node[anchor=south]{yes} (pro1);
\path [line] (pro1) -- (func1);
\path [line] (func1) -- (return1);

\path [line] (dec1) --node[anchor=east]{no}  (dec2);
\path [line] (dec2) -- node[anchor=south]{yes} (pro2);
\path [line] (pro2) -- (return2);
\path [line] (dec2) -- node[anchor=east]{no} (return3);

\end{tikzpicture}

\clearpage
\section{Flowchart for setValues \& calculateValues\_ functions in \DuMuX}
% \caption {Flow chart for \code{setValues} and \code{calculateValues\_} functions}
\begin{tikzpicture}[node distance = 2cm, auto]
    % Place nodes
    \node [cloud] (init) {\code{setValues} function};
    
    \node [block, below of=init, yshift=-0.5cm] (identify) {Loop over all control volumes};
    
    \node [cloud, below of=identify, yshift=-1.2cm] (evaluate) {call \code{calculateValues\_} function that calculates 
    pH, $\mathrm{m^{Ca^{2+}}}$, $\mathrm{m^{TIC}}$, $\mathrm{m^{CO_3^{2-}}}$, 
    $\mathrm{r_{diss}}$, and Omega , store in a vector and return it};
    
    \node [cloud, right of=evaluate, xshift=6cm, yshift = 5.8cm] (expert) {\code{calculateValues\_()} function};
    \node [block, below of=expert] (pro1) {Extract pH stored in its vector and calculate \[\mathrm{m^{H^+} = 10^{-pH}}\]};
    
    \node [block, below of=pro1, yshift=-0.8cm](pro2) {Get the values $\mathrm{m^{Ca^{2+}}}$ and $\mathrm{m^{TIC}}$ 
    from \code{volVars} and convert it to molality};
    
    \node[block, below of=pro2, yshift=-1cm] (pro3) {call \code{brentAlgorithm\_} function to solve charge balance 
    equation (\ref{eq:chargebalfinal}) for $\mathrm{m^{H^+}}$};
    
    \node[block, below of=pro3, yshift=-0.5cm] (pro4) {Calculate $\mathrm{m^{CO_3^{2-}}}$ = $\frac{\ce{m^{TIC}}}
    {\frac{\left(\ce{m^{\ce{H^{+}}}}\right)^2}{\ce{k_{diss,1}} \cdot \ce{k_{diss,2}}}+1+\frac{\ce{m^{\ce{H^{+}}}}}{\ce{k_{diss,2}}}}$};
    
    \node[block, below of=pro4, yshift=-0.5cm] (pro5) {Calculate Omega = $\mathrm{m^{Ca^{2+}}}$ * $\mathrm{m^{CO_3^{2-}}}$/$\mathrm{K_{sp}}$.};
    
    \node[block, below of=pro5, yshift=0cm] (pro6){Initialize \ce{r_{diss}} = 0};
    
    \node[decision, below of=pro6, yshift=-1cm] (dec1) {Omega $<$ 1.0};
    
    \node[block, left of=dec1, xshift=-7cm] (pro7) {$\mathrm{r_{diss}}$ = $\mathrm{k_{diss,1} * m^{H^+} + k_{diss,2} * (1 - Omega)}$.};
    
    \node [block, below of=pro7, yshift=-1.5cm] (pro9) {return pH, $\mathrm{m^{Ca^{2+}}}$, $\mathrm{m^{TIC}}$, $\mathrm{m^{CO_3^{2-}}}$, 
    $\mathrm{r_{diss}}$, and Omega};
    
    \node [block, below of=evaluate, node distance=3.5cm] (decide) {extract the values from the returned vector and save it in 
    their respective vectors};

    % path edges
    \path [line] (init) -- (identify);
    \path [line] (identify) -- (evaluate);
    \path [line] (evaluate) -- (decide);
    \path [line,dashed] (evaluate.east) -- ++ (0.5cm,0) |- (expert.west);
    \path [line] (expert) -- (pro1);
    \path [line] (pro1) -- (pro2);
    \path [line] (pro2) -- (pro3);
    \path [line] (pro3) -- (pro4);
    \path [line] (pro4) -- (pro5);
    \path [line] (pro5) -- (pro6);
    \path [line] (pro6) -- (dec1);
    \path [line] (dec1) |- node[anchor=west] {no} (pro9);
    \path [line] (pro7) -- (pro9);
    \path [line] (dec1) -- node[anchor=south]{yes} (pro7);
\end{tikzpicture}


\endinput
