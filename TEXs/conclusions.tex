\chapter{Conclusions} \label{chapter:conclusions}
\thispagestyle{empty}
We developed a numerical model in \DuMuX, also validated successfully with \MATLAB, that accounts for calcite dissolution 
in cave surfaces as a source/sink term in the Navier-Stokes model. The numerical 
comparison between \DuMuX and \MATLAB showed very good consistency.\\

We saw in chapter \ref{chapter:results} that the rate of calcite dissolution, pH and the 
concentration of $\mathrm{m^{Ca^{2+}}}$, $\mathrm{m^{TIC}}$, $\mathrm{m^{CO_3^{2-}}}$, $\mathrm{r_{diss}}$ strongly depends on 
flow-velocity and the concentration gradient between the region above and below the epiphreatic karst water table that 
acts as the limiting factors for karstification. \\

The developed model did not consider some scenarios prevalent in a bigger geological system. It did not consider the gravity 
as we set a constant hydrostatic pressure throughout the domain; the change in density of the karst water due to the addition of total inorganic carbon 
and calcium due to calcite dissolution; and a seasonal fluctuation of the \ce{CO2} concentration above the epiphreatic karst water table, 
rather we set a constant concentration at the top boundary. The concentration of \ce{CO2} varies throughout the year, being highest during summers and lowest during 
winters (\cref{tab:CO2fluctuations}). This variation will have a significant impact on density-driven fingering on calcite dissolution. 
We saw grid grading did not matter in our model but in a bigger geological system, this could play a role in resolving the boundary layer 
developed near the reactive wall. The \DuMuX model only had reactive source/sink terms at the cells on the wall. 
To closely model a cave scenario, the reactive source/sink terms should also be solved throughout the domain. \\

With gravity and compositional, compressible systems, a problem may arise if pressure is set at the boundary as a boundary condition, 
which we did in our model, instead of setting the actual hydrostatic pressure calculated considering the density of the fluid and its composition. 
The pressure profile developed due to the change in density and composition of the fluid might not match with the 
user-defined profile that causes unphysical spurious in/out-flows. 
In our \DuMuX model, we set a constant pressure of 1e5 Pa throughout the domain. So if the gravity was turned on, this would lead to a different 
pressure profile in the fluid inside the domain, which would then cause undesirable in/out-flows depending on the pressure 
difference. \\

Expanding the model to account for the aforementioned limitations would be the next step in exploring density-induced fingering 
of \ce{CO2}-enriched water as a mechanism for speleogenesis. This would be challenging especially with the numerical overhead 
that comes with solving Navier-Stokes model with the reactive source/sink terms that will significantly increase the computational time. 
A thorough literature review would be essential to consider the density of karst water due to the addition of total inorganic carbon and calcium due to 
calcite dissolution.


\endinput
