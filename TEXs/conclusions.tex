\chapter{Conclusions} \label{chapter:conclusions}
\thispagestyle{empty}
We were able to develop a numerical model in \DuMuX that accounts for the dissolution rates in cave surfaces as a source/sink term in the Navier-Stokes model. \\


Numerical comparison between \DuMuX and \MATLAB shows a satisfactory consistency, especially during initial few time steps. The numerical solvers used to solve the non-linear equation was different: we used \code{vpasolve} function in \MATLAB and Newton-Raphson algorithm in \DuMuX. Not only that but also the concept to consider the domain size was different. We used a batch volume in \MATLAB where the mixing would happen instantly as the dissolution proceeds -- we only had a single cell, whereas in \DuMuX it is influenced by the diffusion of dissolved calcium-carbonate. We could not avoid these differences in the model in these platforms; however, we did make sure the domain size and the boundary and initial conditions were the same and that the process would be comparable during first few initial time steps where the local truncation error would not have a considerable impact on the global error in estimating the primary variables between these two numerical models.\\

We saw in the chapter \ref{chapter:results} that the steady-state time and the primary variables would change with changing scenarios. Initial pH, concentration of \ce{CO2} at the top of the cave, initial concentration of \ce{CO2} in the karst water, flow velocity are the limiting factors that influences the time-scale for the karstification mechanism.\\

We were not able to predict the time scale for the real–world scenario which requires bigger domain, gravity cannot be switched off, concentration of calcium ions should be considered for the calculation of density, seasonal variation of \ce{CO2} at the top should be considered. Grid grading matters in particular near the reactive wall to resolve the developed boundary layer. The problem with gravity and compositional, compressible systems may arise if a pressure boundary is set on a boundary face, where "true" hydrostatic pressure should be set. "True" hydrostatic pressure depends on the density of the fluid which in turn depends on the composition of it. So, if a fixed pressure value or profile is set, considering a change in density and composition of the fluid, spurious in- or out-flows due to non-matching pressure distribution might occur. In our \DuMuX model, we set a constant pressure of 1e5 Pa throughout the domain. So, when the gravity is turned on, this will lead to a different pressure profile in the fluid inside the domain, which causes an undesirable in/out-flows depending on the pressure difference. Further, we fixed the top boundary concentration of \ce{CO2} throughout the simulation, which in-fact is seldom true in real-world scenario. The \ce{CO2} varies throughout the year with highest during summers and lowest during winters, which will have a significant impact in the process. A proper model that accounts for the density change due dissolution of calcium-carbonate is desirable to closely estimate the density of the karst water.\\

Expanding the model to account for these limitations would be the next step in exploring density-induced fingering of \ce{CO2}-enriched water. This would be challenging, especially with the numerical overhead that comes with solving the chemical equations that would significantly increase the computational time. A through literature review would be essential to consider the density change due to calcite dissolution in karst water. \\


\endinput
