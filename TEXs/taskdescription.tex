\chapter*{Task Description}
% \pdfbookmark[1]{Task Description}{Task Description}
\thispagestyle{empty}

Density-induced fingering due to dissolution of \ce{CO2} in water is currently considered by 
a small group of researchers a potential mechanism for speleogenesis in the phreatic zone of a cave. 
This mechanism is so far not discussed in current literature. \citet{Scherzer2017} denoted it as 
nerochytic speleogenesis(NERO).\\
\citet{Class2020} have shown in an experimental and numerical simulation study that typical fingering 
velocities should be expected in the order of less than a centimeter per minute, dependent of course on 
the difference in \ce{CO2} concentration between the cave air and the water. \\

This thesis is aiming at developing conceptual ideas for modeling the dissolution of calcite surfaces in 
a cave which are exposed to this type of fingering. Since this phenomenon is currently not described in 
the literature, it is necessary to do an intensive literature study on related research, in particular 
publications in the cave and speleogenesis communities who worked on karstification due to water flow and 
who described some kind of kinetic
calcite dissolution models.\\

The goal of this thesis is to develop a model that includes the required chemical components: \ce{CO2} related to 
the dissolved carbonate via pH, Ca-Ions. This model can then be implemented in the 
numerical simulator \DuMuX \citep{Koch2020}, and coupled to the available Navier-Stokes model. 
The scenario to be modeled will be a small 2D model domain where the immediate vicinity of a calcite 
surface will be modeled to estimate calcite dissolution rates over time dependent on \ce{CO2} 
concentrations and velocities. The size of the domain could limit the resolution of a boundary-layer developed at the wall.

\endinput
