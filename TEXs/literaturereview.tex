\chapter{Literature review}\label{chapter:LiteratureReview}
\thispagestyle{empty}

Density-driven dissolution of \ce{CO2} is not unknown and is widely described in research articles as 
geological sequestration, e.g.\cite{lindeberg1997reservoir, bachu2007co2} and is denoted as solubility 
trapping \cite{metz2005carbon}. However, it is seldom considered as a mechanism for speleogenesis. 
We are aiming at developing a free-flow Navier-Stokes model with a reactive source/sink term and pointing 
out the limiting factors for the rate of dissolution of calcite. This literature review gives a 
brief overview of the well-established findings and highlights gaps in literature.

\section{Dreybrodt model}\label{sec:dreybrodt}
\citet{Dreybrodt2012}, in his book, describes the process karstification where he states a circulation of water 
controls the development of karst. According to \citet{Dreybrodt2012}, once suitable pathways exist for the 
surface water to penetrate the micro-millimeters sized (in the order of several 10$\mu$) fissures, surface 
water starts to dissolve the limestone by carbonic-acid containing water along its way enlarging the primary fissures 
and increasing the amount of water transported through the limestone. This creates a cascading effect which leads 
to progressive enlargement of fissures and creation of new pathways in an aquifer. His study was for stagnant water 
and does not account for a system having background-flow.\\

In his model \cite{Dreybrodt2012}, he assumed a boundary layer of thickness $\delta$ immediate with the rock surface 
where, due to dissolution, exchange of flux of \ce{Ca^{2+}}, \ce{CO3^{2-}}, and \ce{HCO3^{-}} ions between the solid 
surface and the bulk occurs. \ce{CO2} diffuses into the bulk from the atmosphere forming \ce{CO2}-rich water with a 
higher density than water triggers protruding fingers into the bulk \cite{Class2020}. \citet{Plummer1978} determined 
the rate of dissolution using an equation, also known as, Plummer-Wigley-Parkhurst (PWP) Equation which could be coupled 
with continuity equation \ref{eq:contiEq} in an attempt to solve for the fluxes formed due to the dissolution of calcite.

\newpage
\paragraph*{Plummer-Wigley-Parkhurst Equation (PWP)}\mbox{}\\

The chemical reaction on the surface of the limestone: \\
\ce{CaCO3 + H2O + CO2 -> CaCO3 + H2CO3 -> Ca^{2+} + 2HCO3^{-}}\\
This is a global reaction at the calcite surface on a \ce{H2O-CO2} system and it comprises of three different reactions.
\begin{flalign} \label{eq:CalciteDiss}
     \ce{CaCO3 + H^+ <=> Ca^{2+} + HCO3^-};  \\
     \ce{CaCO3 + H2CO3 <=> Ca^{2+} + 2HCO3^-};  \\
     \ce{CaCO3 + H2CO3 <=> Ca^{2+} + CO3^{2-} + H2O .} 
\end{flalign}

Further, \ce{CO2} reacts with water to form carbonic acid.
\begin{flalign} \label{eq:CarbonicAcidForm}
     \ce{H2O + CO2 <=> H2CO3}, 
\end{flalign}

\ce{H2CO3} dissociates into \ce{H^+} and \ce{HCO3^-}:
\begin{flalign} \label{eq:CarbonicAcidDiss}
     \ce{H2CO3 <=> H^+ + HCO3^-}, 
\end{flalign}

and \ce{HCO3^-} dissociates into:
\begin{flalign} \label{eq:BicarbonateDiss}
     \ce{HCO3^- <=> H^+ + CO3^{2-}}. 
\end{flalign}

The net rate of dissolution (\ce{r_{diss}}) is given by an equation \cite{Plummer1978}:

\begin{equation}\label{eq:PWP} % Give a unique label
 \ce{r_{diss}} = k_1a_{H^+} + k_2a_{\ce{H2CO3}}^* + k_3a_{\ce{H2O}} - k_4a_{\ce{Ca^{2+}}}a_{\ce{HCO3^-}}
\end{equation}  

The rate constants: \ce{k1, k2, k3, and k4} are the first order rate constants which depends on temperature -- except for \ce{k1} 
which depends on temperature and partial pressure of \ce{CO2} ($P_{\ce{CO2}}$ -- and can be predicted as per \cite{Plummer1978}. 
By using these rate constants and the fluxes, rate of dissolution can be calculated. \\
\citet{Plummer1978} defines these rate constants as: (where T in K and $k_x$ in cm/sec)

\begin{flalign}\label{eq:k1}
\log k_1 = 0.198 - 444/T 
\end{flalign}

\begin{flalign}\label{eq:k2}
\log k_2 = 2.84-2177/T 
\end{flalign} 

At temperature less than $25^\circ$C:
\begin{flalign}\label{eq:k3under25} % Give a unique label
\log k_3 = -5.86 -317/T 
\end{flalign} 

At temperature more than $25^\circ$C:
\begin{flalign}\label{eq:k3over25} % Give a unique label
\log k_3 = -1.10 - 1737/T 
\end{flalign} 

\begin{equation}\label{eq:k4} % Give a unique label
k_4 = \frac{K_2}{K_e} \left\{ {k^{'}}_1 + \frac{1}{a_{{{H^{+}}}_{(s)}}}\left[k_2a_{{\ce{H2CO3}^{*}}_{(s)}} + k_3a_{{\ce{H2O}}_{(s)}}\right] \right\}
\end{equation}
where $K_2$ and $K_e$ are the equilibrium constants for the second dissociation of carbonic acid and calcite, 
respectively, ${k^{'}}_{1}$ is the forward rate constant for \ce{CaCO3 + H^+ <=> Ca^{2+} + HCO3^-} (${k^{'}}_{1}$ is 10-20 
times larger than $k_1$), and the subscript \say{s} stands for adsorption surface values.

\section{Summary}\label{sec:summary}
\paragraph*{Dreybrodt and Gabrov{\v{s}}ek's work} \citet{gabrovvsek2000role} discusses two cave forming mechanisms: 
Mixing Corrosion, explained by \citet{bogli1980physical}, in combination with the linear dissolution kinetics; 
and non-linear dissolution kinetics generating extended karst conduits. \citet{gabrovvsek2000role} combines both 
these approaches to study evolution of karst aquifers. \citet{Dreybrodt1996} estimates the dissolution rates for 
a system of \ce{H2O-CO2-CaCO3}.

\begin{equation}
    F(c) = k_n(1-c/c_{eq})^n
\end{equation}

where $k_n$ is a constant [$\mathrm{mol/cm^2s}$], $c$ is the actual concentration of dissolved calcium, and $c_{eq}$ is its 
equilibrium concentration [$\mathrm{mol/cm^3}$].\\
For linear dissolution rate $n = 1$, development of karst channel takes geologically unrealistic times as the law 
breaks down as the penetrating water goes into a deep rock; with an increase in distance, the dissolution rate drops 
exponentially \cite{Dreybrodt1996}. Hence, the law only accounts for entrance widening. 
\citet{dreybrodt2004dissolution}, \citet{gabrovvsek2000role} suggest the concept of mixing corrosion 
proposed by \citet{bogli1980physical} together with non-linear rate-laws with $n$ varying between 3 to 6 for $c > c_{eq}$, 
which helps in understanding cave evolution in deep rocks. Mixing corrosion alone could not explain cave formation developed 
along the bedding planes without intersecting by joints \cite{ford1978development}; thus, \citet{gabrovvsek2000role} propose 
a non-linear rate law to bridge the gap. \citet{gabrovvsek2000role} makes it evident in a model, where they assume \ce{CO2} 
could not be replenished in deep, micro-millimeters sized fissures once it's being consumed during dissolution. As $c$ 
approaches $c_{eq}$, the reaction kinetics are inhibited as per non-linear kinetics, which in turn saves its remaining 
dissolution power and takes it deeper into the rock. \citet{gabrovvsek2000role} further argue that mixing corrosion is not 
a prerequisite for the evolution of caves, but together with non-linear dissolution rates law, it could contribute to karst 
evolution. Mixing corrosion \cite{bogli1980physical} is crucial in early karst development, but it has little importance in 
mature karst. \\



\citet{bonacci2001analysis} states the karst systems has complex drainage systems and karstification is enhanced by biologically 
active soil layers enriched with \ce{CO2}. His study still does not account for the system with no background flow/small 
where \ce{CO2} is replenished. \\

\citet{bakalowicz2005karst} describes climate is enhancing karstification. Groundwater flow carries the produced \ce{CO2} by 
geological activity in soil or at a depth by geological processes. Pipe flow condition prevails under pressure or at the 
atmospheric condition which also transports the produced products away. Karstification can be very rapid in terms of geological 
times, a few thousand years is sufficient to build an integrated karst network.\\

\citet{mangin1975contribution} defines the concept of Potential for Karst Development (PKD) which determines the flux of solvent 
through the rock and is driven by the amount of precipitation, the soil partial pressure of \ce{CO2} -- thus, the biological activity 
and weather conditions play a role -- and the hydraulic gradient between the karst recharge area and karst spring level. \\

\citet{mohammadi2007method}, also refer to PKD concept of \citet{mangin1975contribution}, state the dissolution potential results from 
the carbonic-acid containing water, but they do not mention how it was dissolved in the first place. \\

A relevant case study with respect to ours carried out by \citet{atkinson1977carbon} emphasizes the importance of a 
system being closed and opened to the atmosphere in karstic developments. In an open system, gaseous \ce{CO2} is dissolved into the water 
forming carbonic-acid which leads to replenishment of dissolution potential until the steady-state 
is reached, whereas in the closed system, dissolved \ce{CO2} in karst water provokes the dissolution until all the carbonic acid is depleted, 
hence system comes to internal equilibrium -- unlike in an open system, replenishment does not occur in a closed system. 
Seasonal fluctuations and mean value of \ce{CO2} concentrations in different soils are presented in table \ref{tab:CO2fluctuations}, 
which could play a role while modeling a geological system. \\

\begin{table}[ht]
\small\addtolength{\tabcolsep}{-5pt}
\centering
\caption [Published results of carbon dioxide concentration in soil air \cite{white2018karst}] {\textbf{Published results of carbon dioxide concentration in soil air \cite{white2018karst}}}
\begin{tabular}{lccccc}
    \hline
    Soil/vegetation & \multicolumn{4}{c}{Percent carbon dioxide} & Source\\
    \cline{2-5}
    & usual & summer & winter & extreme values &\\
    \hline
    %Soil/vegetation              usual             summer         winter       extreme    source    
    Arable                       & 0.9         &              &              &             & \cite{russell1973soil} \\
    Pasture                      & 0.5--1.5    &              &              & 0.5--11.5   & \cite{russell1973soil} \\
    Sandy arable                 & 0.16        &              &              & 0.05--3.0   & \cite{russell1973soil} \\
    Arable loam                  & 0.23        &              &              & 0.07--0.55  & \cite{russell1973soil} \\
    Moorland                     & 0.65        &              &              & 0.28--1.4   & \cite{russell1973soil} \\
    Arable                       & 0.1--0.2    &              &              & 0.01--1.4   & \cite{russell1973soil} \\
    Manured arable               & 0.4         &              &              & 0.03--3.2   & \cite{russell1973soil} \\
    Grassland                    & 1.6         &              &              & 0.3--3.3    & \cite{russell1973soil} \\
    Dark chestnut: 7cm           & 0.1         &              &              &             & \cite{chulakov1959} \\
    \hspace{27mm} 300cm          & 1.7         &              &              &             & \cite{chulakov1959} \\
    Steppe: trees                & 2.5--3.4    &              &              &             & \cite{matskevitch1957} \\
    \hspace{14mm} herbaceous     & 1.2--2.0    &              &              &             & \cite{matskevitch1957} \\
    Sandy loam 30cm              &             & 2.5          & 0.3          & 0.2--3.6    & \cite{gerstenhauer1969offene} \\
    Sandy loam 30cm              &             & 1.5          & 0.1          & 0.1--1.9    & \cite{gerstenhauer1969offene} \\    
    Loamy sand 50cm              &             & 0.8          & 0.2          & 0.2--1.1    & \cite{gerstenhauer1969offene} \\
    \hspace{22mm} 20cm           &             & 0.9          & 0.1          & 0.05--2.0   & \cite{gerstenhauer1969offene} \\
    Brown earth                  & 0.27--0.41  &              &              & 0.08--0.7   & \cite{nicholson1969new} \\
    Orchard/grass 30cm           &             & 1.5--2.5     & 0.1--1.0     &             & \cite{boynton1944normal} \\
    \hspace{26mm} 90cm           &             & 2--5         & 1--3         &             & \cite{boynton1944normal} \\
    \hspace{26mm} 150cm          &             & 4--9         & 2--6         &             & \cite{boynton1944normal} \\
    Valley bog 5cm               &             & 1--3.5       &              &             & \cite{sheikh1969responses} \\
    Mean values                  & 0.9         & 2.5          & 1.0          & 0.17--2.8   &   \\    \hline
\end{tabular}
\label{tab:CO2fluctuations}
\end{table}

\citet{garcia2011numerical} modeled coastal aquifer karst process with a Darcy model and stated that once the mixing 
zone occurs for sea and freshwater is established; freshwater, which is under saturated with calcite, and with the continuous inflow 
of rainwater drives the carbonate dissolution mechanism longer and prevents the system in reaching steady-state. 
This process might last decades until the system comes to a "pseudo" steady-state.\\

\citet{gulley2014vadose} states the driving force for the dissolution of calcite is the fluctuations in $p_{\ce{CO2}}$ which is the 
same as we would want in our model, but they did not explain the mechanism of density-driven dissolution. They assume flow 
takes place because of pressure difference in \ce{CO2}, which adds \ce{CO2} in the karst water and thereby increases the dissolution potential.

\section{Conclusion}\label{sec:conclusionLiterature}
As relevant as their (\citet{Dreybrodt1996}, \citet{gabrovvsek2000role}, \citet{dreybrodt2004dissolution}, \citet{Dreybrodt2012}) 
research among others, there is a considerable gap in considering density-driven dissolution as a mechanism for speleogenesis. 
Together with the \citet{gabrovvsek2000role} and \citet{bogli1980physical} mechanisms in transport and karstification, 
we would like to propose a third explanation on how the dissolution potential could reach below the epiphreatic karst water table 
and deeper into the rock. We adhere to the idea that solving chemical kinetics for calcite dissolution interacting with density 
induced fingering regime could potentially be the first step in substantiating the concept which could then be coupled with the 
model developed by \citet{Class2020}.
\endinput
