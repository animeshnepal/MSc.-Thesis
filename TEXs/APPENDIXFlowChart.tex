\chapter{APPENDIX: Flowchart}\label{app:appendixA}
\thispagestyle{empty}


\section{Flow chart that shows the main program flow}
\centering
\begin{tikzpicture}[node distance=1.7cm]

\node (main) [startstop] {Main};

\node (func3) [predefinedprocess, below of=main] {Set initial and boundary conditions for the problem};

\node (pro12) [process, below of=func3] {Initialize solution vector as per Navier-Stokes model};

\node (in1)[io, below of=pro12] {Import time loop parameters: tEnd, maxDt and dt};

\node (pro1)[process, below of=in1,yshift=-0.5cm] {Add fields such as \code{pH}, \code{mCa}, \code{mTIC}, \code{mCO2}, \code{rdiss} and \code{Omega} for the output in the vtk file};

\node (pro2)[process, below of=pro1,yshift=-1.0cm] {Initialize two csv file streams (\code{fout} and \code{fnormout}) for normalized and actual simulation time having above mentioned variables as their columns};

\node (pro3)[process, below of=pro2,yshift=-1.0cm] {Initialize \code{time} variable that keeps track of the total time elapsed in the simulation};

\node (pro4)[process, below of=pro3,yshift=-0.7cm] {Initialize \code{normTimeFlag} to \code{true} which checks for normalized pH value, which in this case set to 7.0 };

\node (func1) [predefinedprocess, below of=pro4, yshift=-1.0cm] {call \code{setValues} function which calculates and sets \code{pH}, \code{mCa}, \code{mTIC}, \code{mCO2}, \code{rdiss} and \code{Omega} throughout the domain in a given time step};



\node (conn1)[connector, below of=func1, yshift=-1.0cm] {A};
\node (conn2)[connector, right of=conn1, xshift=5cm] {B};
 
% lines
\path [line] (main) -- (func3);
\path [line] (func3) -- (pro12);
\path [line] (pro12) -- (in1);
\path [line] (in1) -- (pro1);
\path [line] (pro1) -- (pro2);
\path [line] (pro2) -- (pro3);
\path [line] (pro3) -- (pro4);
\path [line] (pro4) -- (func1);
\path [line] (func1) -- (conn1);
\path [line] (conn2) |- (0,-16.3);
\end{tikzpicture}


\begin{tikzpicture}[node distance=2cm]
\node (conn2)[connector] {A};
\node (func4) [predefinedprocess, below of=conn2, yshift=-1.0cm] {Solve the continuity equation \[\frac{\partial (X\textsuperscript{$\kappa$})}{\partial t} 
 + \nabla\cdot(\textbf{v}X\textsuperscript{$\kappa$} - D\textsuperscript{$\kappa$}\nabla X\textsuperscript{$\kappa$}) = q\]; call \code{reactionSource} function to add source and sink terms};
\node (conn3)[connector, right of=conn2, xshift=7cm] {B};
\node (func2) [predefinedprocess, below of=func4, yshift=-1.0cm] {retrieve the value of pH Vector by calling \code{getpH} function};

\node (dec1) [decision, below of=func2, yshift=-1.5cm]{If \code{pH} $>$ 7.0 \&\& \code{normTimeFlag}};

\node (pro5)[process, right of=dec1,xshift=4cm, text width=4cm] { set \code{normTimeFlag} to \code{false} and note the \code{time}};

\node (pro6)[process, below of=dec1, yshift=-1.7cm] {retrieve values of \code{pH}, \code{mCa}, \code{mTIC}, \code{mCO2}, \code{rdiss} and \code{Omega} by calling respective getter function};

\node (pro7)[process, below of=pro6] { write the output values to a vtk file};

\node(pro8) [process, below of=pro7] {update the \code{time} variable};

\node (conn4)[connector, below of=pro8] {C};

\node (conn4a)[connector, right of=conn4, xshift=7cm] {B};

\path [line] (conn2) -- (func4);
\path [line] (func4) -- (func2);
\path [line] (func2) -- (dec1);
\path [line] (dec1) -- node[anchor=south]{yes} (pro5);
\path [line] (dec1) -- node[anchor=east]{no} (pro6);
\path [line] (pro5.south) |- (0,-11.9);
\path [line] (pro6) -- (pro7);
\path [line] (pro7) -- (pro8);
\path [line] (pro8) -- (conn4);
\path [line] (conn4a) -- (conn3);
\end{tikzpicture}



\centering
\begin{tikzpicture}[node distance=2cm]
\node (conn5)[connector] {C};
\node (conn5a)[connector, right of=conn5, xshift=5cm] {B};
\node(dec2) [decision, below of=conn5, yshift=-2.0cm, text width= 4.2cm] {check for \code{timeLoop->finished()}?};

\node(pro9) [process, below of=dec2, yshift=-2.0cm] {close filestream \code{fout}};
\node(pro10) [process, below of=pro9] {open the filled table \code{tableDUMUX.csv} which has \code{Time}, \code{pH}, \code{mCa}, \code{mTIC}, \code{mCO2}, \code{rdiss} and \code{Omega} as its column entries};

\node(pro11) [process, below of=pro10, yshift=-1.2cm] {Divide the time entries in the column \code{Time} by the value \code{normTime} to get the normalized time and assign the values back to filestream \code{fnormout} which will save the values in \code{normTableDUMUX.csv}};

\node (process) [process, below of=pro11, yshift=-1cm] {close filestream \code{fnormout}};

\node(end)[startstop, below of=process] {End};

% lines

\path [line] (conn5) -- (dec2);
\path [line] (dec2) -| node[xshift=-0.5cm, yshift=0.3cm]{no} (conn5a);
\path [line] (dec2) -- node[anchor=east]{yes} (pro9);
\path [line] (pro9) -- (pro10);
\path [line] (pro10) -- (pro11);
\path [line] (pro11) -- (process);
\path [line] (process) -- (end);

\end{tikzpicture}

\section{Flow chart for Initial and Boundary conditions}
\begin{tikzpicture}[node distance=2cm]
\node (func1) [predefinedprocess] {Initial and Boundary conditions};

\node (dec1) [decision, below of=func1, yshift=-1.5cm, text width= 2.6cm] {if \code{onTopBoundary}?};
\node (pro1) [description, right of=dec1, xshift=5cm] {set Dirichlet conditions for velocities};
\node (dec2) [decision, below of=pro1, yshift=-0.8cm, text width = 2cm] {if \code{!NOFLOW}?};
\node (pro2) [description, below of=dec2, yshift=-0.8cm] {set Dirichlet conditions for Calcium and C0\textsubscript{2} concentrations};
\node (pro3) [description, right of=dec2, xshift=2.0cm, text width= 3.0cm] {set Neumann no flow for Calcium and C0\textsubscript{2} fluxes};
\node (return1) [description, below of=pro2,text width= 3.0cm] {return \code{values}};


\node (dec3) [decision, below of=dec1, yshift=-8cm, text width= 3.2cm] {if \code{onBottomBoundary}?};
\node (pro4) [description, right of=dec3, xshift=5cm] {set Dirichlet conditions for pressure};
\node (dec4) [decision, below of=pro4, yshift=-0.8cm, text width = 2cm] {if \code{!NOFLOW}?};
\node (pro5) [description, below of=dec4, yshift=-0.8cm] {set Outflow for Calcium and C0\textsubscript{2} fluxes};
\node (pro6) [description, right of=dec4, xshift=2.0cm, text width= 3.0cm] {set Neumann no flow for Calcium and C0\textsubscript{2} fluxes};
\node (return2) [description, below of=pro5, text width= 3.0cm] {return \code{values}};

\node (conn1) [connector, below of=dec3, yshift=-6cm] {A};

% lines
\path [line] (func1) -- (dec1);
\path [line] (dec1) -- node[anchor=south]{yes} (pro1);
\path [line] (pro1) -- (dec2);
\path [line] (dec2) -- node[anchor=east]{yes} (pro2);
\path [line] (dec2) -- node[anchor=south]{no} (pro3);
\path [line] (pro2) -- (return1);
\path [line] (pro3) |- (return1);

\path [line] (dec1) -- node[anchor=east]{no}  (dec3);
\path [line] (dec3) -- node[anchor=south]{yes} (pro4);
\path [line] (pro4) -- (dec4);
\path [line] (dec4) -- node[anchor=east]{yes} (pro5);
\path [line] (dec4) -- node[anchor=south]{no} (pro6);
\path [line] (pro6) |- (return2);
\path [line] (pro5) -- (return2);
\path [line] (dec3) -- node[anchor=east]{no}  (conn1);

\end{tikzpicture}


\begin{tikzpicture}[node distance=2cm]
\node (conn1) [connector] {A};
\node (dec1) [decision, below of=conn1, text width= 2.8cm, yshift=-1.0cm] {if \code{onLeftBoundary}?};
\node (pro1) [description, right of=dec1, xshift=6cm] {set Dirichlet conditions for velocities};
\node (func1) [predefinedprocess, below of=pro1, text width=5cm, yshift=-0.5cm] {set Neumann conditions for Calcium and C0\textsubscript{2} fluxes by calling \code{reactionSource} function};
\node (return1) [description, below of=func1, text width= 3.0cm, yshift=-0.5cm] {return \code{values}};

\node (dec2) [decision, below of=dec1, yshift=-5.0cm, text width= 3.0cm] {if \code{onRightBoundary}?};
\node (pro2) [description, right of=dec2, xshift=6cm] {set symmetric property};
\node (return2) [description, below of=pro2, text width= 3.0cm] {return \code{values}};

\node (return3) [description, below of=dec2, text width= 3.0cm, yshift=-2cm] {return \code{values}};

% lines
\path [line] (conn1) -- (dec1);
\path [line] (dec1) --node[anchor=south]{yes} (pro1);
\path [line] (pro1) -- (func1);
\path [line] (func1) -- (return1);

\path [line] (dec1) --node[anchor=east]{no}  (dec2);
\path [line] (dec2) -- node[anchor=south]{yes} (pro2);
\path [line] (pro2) -- (return2);
\path [line] (dec2) -- node[anchor=east]{no} (return3);

\end{tikzpicture}

\clearpage
\section{Flow chart for {\fontfamily{pcr}\selectfont \textbf{setValues}} \& {\fontfamily{pcr}\selectfont\textbf{calculateValues\_}} functions}
% \caption {Flow chart for \code{setValues} and \code{calculateValues\_} functions}
\begin{tikzpicture}[node distance = 2cm, auto]
    % Place nodes
    \node [cloud] (init) {\code{setValues} function};
    
    \node [block, below of=init, yshift=-0.5cm] (identify) {Loop over all the \code{scv}};
    
    \node [cloud, below of=identify, yshift=-1.2cm] (evaluate) {call \code{calculateValues\_} function, which calculates \code{pH}, \code{mCa}, \code{mTIC}, \code{mCO2}, \code{rdiss} and \code{Omega} in a given time for the \code{scv}};
    
    \node [cloud, right of=evaluate, xshift=6cm, yshift = 5.8cm] (expert) {\code{calculateValues\_()} function};
    \node [block, below of=expert] (pro1) {initialize \code{mCa} by extracting values from \code{pHValues\_} vector using index \code{scv.dofIndex()}};
    
    \node [block, below of=pro1, yshift=-0.8cm](pro2) {retrieve the values \code{mCa, mTIC} from \code{volVars} using \code{molefraction} member function and convert the value to molality by calling \code{moleFracToMolality\_} function};
    
    \node[block, below of=pro2, yshift=-1cm] (pro3) {call \code{localNewton\_} function to solve for \code{mH} from a non-linear equation in \code{mH}. The new value of \code{mH} would be the value for the current time step};
    
    \node[block, below of=pro3, yshift=-0.5cm] (pro4) {calculate \code{mCO3}\[  = \frac{mTIC}{\frac{mH^2}{k1*k2} + \frac{mH}{k2} + 1}\]};
    
    \node[block, below of=pro4, yshift=-0.5cm] (pro5) {calculate \code{Omega} = \[\frac{mCa * mCO3}{Ksp}\]};
    
    \node[block, below of=pro5, yshift=0cm] (pro6){initialize \code{rdiss} = 0};
    
    \node[decision, below of=pro6, yshift=-1cm] (dec1) {if \code{Omega} $<$ 1.0};
    
    \node[block, left of=dec1, xshift=-7cm] (pro7) {\code{rdiss} = (\code{kdiss1 * mH + kdiss2) * (1-Omega)}};
    
    \node[block, below of=dec1, yshift=-2.0cm] (pro8){initialize and resize \code{returnVector} to 6};
    
    \node [block, left of=pro8, xshift=-7cm] (pro9) {assign the calculated values \code{pH}, \code{mCa}, \code{mTIC}, \code{mCO2}, \code{rdiss} and \code{Omega} to the \code{returnVector} and return the the vector};
    
    \node [block, below of=evaluate, node distance=3.5cm] (decide) {extract the values from the \code{returnVector} and save it in its respective index inside vectors using \code{dofIndex()} member function};
    % path edges
    \path [line] (init) -- (identify);
    \path [line] (identify) -- (evaluate);
    \path [line] (evaluate) -- (decide);
    \path [line,dashed] (evaluate.east) -- ++ (0.5cm,0) |- (expert.west);
    \path [line] (expert) -- (pro1);
    \path [line] (pro1) -- (pro2);
    \path [line] (pro2) -- (pro3);
    \path [line] (pro3) -- (pro4);
    \path [line] (pro4) -- (pro5);
    \path [line] (pro5) -- (pro6);
    \path [line] (pro6) -- (dec1);
    \path [line] (dec1) -- node[anchor=west] {no} (pro8);
    \path [line] (pro8) -- (pro9);
    \path [line] (dec1) -- node[anchor=south]{yes} (pro7);
    \path [line] (pro7.south)  |-  (8cm,-20cm);
\end{tikzpicture}


\endinput
